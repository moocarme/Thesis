\begin{center}

Abstract \\
\textsc{Optical Forces Generated by Plasmonic Nanostructures} \\
by \\
\textsc{Matthew Moocarme} \\[0.25in]
\end{center}

\vspace{0.25in}

\noindent Adviser: Professor L.T. Vuong

\vspace{0.25in}

\noindent 
For millennia, scientists have sought to uncover the secrets what holds the world together. From which, the field of optics has been at the forefront, unravelling material properties through investigations of light-matter interactions. 
%Since the birth of the laser in 1960 scientist have been fascinated by the optical phenomena and probed the material properties of many brilliant scientists allowed deep insights into the question in its inmost folds. Step by step we gain more understanding of what light actually is and especially the interaction of light with matter is a treasure trove for new applications and a demanding criterion for the underlying physical theories.
%Since the invention of the laser in 1960, optical scientists have begun to capitalize in earnest on the advances in nanofabrication that is owed to the explosive rise of miniaturized semiconductor electronics. The resulting field, nanophotonics, has opened a vast design space for applied researchers and required revisiting some of the oldest problems and assumptions of optical physics. 
As the field has progressed, the smallest unit at which can be probed and manipulated has subsequently decreased, as such the resulting field of sub-field of optics, nanophotonics, reflecting the processing of light at the nanoscale, has blossomed into a vast design space for both applied and theoretical researchers.
This thesis reflects the advancement of just a few small subtopics in the field, and is divided into an introduction followed by four chapters:

\paragraph{Chapter 1} introduces the field of nanophotonics and explains how electron density waves that propagate along a metal-dielectric interface, commonly known as plasmons, can have such a large impact on the field. Following, is a brief introduction on how plasmonic excitations are born, as well as the forces that result from the high electric field density plasmons are known for.

\paragraph{Chapter 2} investigates magneto-plasmonics, the ability to perturb a plasmonic response with magnetic fields, in gold nanoparticles. Magneto-plasmonics is generally associated with ferromagnetic-plasmonic materials because such optical-derived responses from nonmagnetic materials alone are considered weak, however, this chapter shows the measurable magneto-optical responses in gold nanoparticles. This chapter shows that there exists a transition between linear and nonlinear magneto-optical behaviors in gold nanocolloids that is observable at ultra-low illumination intensities and direct-current magnetic fields. The response is attributed to polarization-dependent nonzero-time-averaged plasmonic loops, vortex power flows, and nanoparticle magnetization. This chapter identifies significant mechanical effects that arise via magnetic-dipole interactions, exhibited via changes in the transmission spectra.


\paragraph{Chapter 3} shows that mechanical forces are produced by plasmonic excitation on nanowires, a phenomenon that has been widely overlooked and points to a new family of plasmonically-driven processes. On a fundamental level, this chapter investigates the action of the Lorentz force on a free electron gas that is bound to the surface of a nanowire. Appreciable mechanical forces are produced by wavelength illuminations between longitudinal and transverse absorption resonances via the excitation of chiral hybrid plasmon modes. The plasmonic activity is associated as the underlying mechanism for nanowire rotation, which explains prior experimental results. The presence of chiral hybrid plasmon modes yield larger net translation and torque forces than either transverse or longitudinal plasmon modes. The asymmetric plasmon behavior affects the complex nonlinear dynamics of plasmonic nonspherical nanoparticles in fluids.


\paragraph{Chapter 4} explores fractal architectures for robust signal transmission. Specifically, the chapter investigates the phenomenon that occurs when plane waves diffract through fractal-patterned apertures, the resulting far-field profiles or diffractals also exhibit iterated, self-similar features. This specific family of architectures enable robust signal processing and spatial multiplexing: arbitrary parts of a diffractal contain sufficient information to recreate the entire original sparse signal.


\paragraph{Chapter 5} presents a metasurface design that produces a chiral response from an array of achiral sub-structures, or meta-atoms. In general, the response of an individual meta-atom is extrapolated to explain the collective response of a metasurface, in a manner that neglects the interactions between meta-atoms. Thus, the convention is that chiral responses are born from metasurfaces composed of chiral meta-atoms. This chapter studies a metasurface that is composed of tilted achiral meta-atoms with no spatial variation of the unit-cell yet derives appreciable optical chirality solely from the asymmetric interactions between meta-atoms. Interactions between meta-atoms are modelled as the Lorentz force that arises from the Larmor radiation of adjacent plasmonic resonators because their inclusion accurately predicts the bonding/anti-bonding modes that are measured experimentally. Experimentally observed are the emergence of multiple polarization eigenmodes, among other polarization-dependent responses, which cannot be modeled with the transmission matrix formalism. The results are essential for the precise characterization and design of metasurfaces. 
