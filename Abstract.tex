\begin{center}
Abstract \\
\textsc{Optical Forces Generated by Plasmonic Nanostructures} \\
by \\
\textsc{Matthew Moocarme} \\[0.25in]
\end{center}

\vspace{0.25in}
\noindent Adviser: Professor Luat T. Vuong
\vspace{0.25in}
\noindent 

For millennia, scientists have sought to uncover the secrets what holds the world together. From which, the field of optics has been at the forefront, unravelling material properties through investigations of light-matter interactions. As the field has progressed, the smallest unit at which can be probed and manipulated has subsequently decreased, as such, the resulting sub-field\textemdash nanophotonics, which reflects the processing of light at the nanoscale, has blossomed into a vast design space for both applied and theoretical researchers.

Plasmonics, the phenomena by which the electron-density of a material oscillates in response to incident electromagnetic radiation, has excited nanophotonics researchers for two reasons. The first is that plasmonics allows for the coupling of light to sub-wavelength dimensions, circumventing the diffraction limit and concentrating electromagnetic fields. The second is that advances in nanofabrication methods, driven by the silicon microelectronics industry, has allowed for the fabrication and development of metallic structures at the nanoscale, a requirement for the excitation of plasmons with visible light.

This thesis explores some aspects of how plasmonics can be used to exploit the design, fabrication and applications of nanostructures that result in materials with highly tailored optical properties. In particular, this thesis will demonstrate the understanding of how high electromagnetic field density that plasmons create produce optically-generated forces. Applications of the optically-generated forces presented here include the advanced control of nanoparticles that form the building blocks of metamaterials, as well as metasurface designs that vary polarization and encode information. 

%Metamaterials and their 2D counterparts\textemdash metasurfaces, are example of applications may offer the potential to create novel materials from invisibility cloaks that hide objects to superlenses that beat the diffraction limit. 

%In the last three or so decades, optical scientists have begun to capitalize in earnest on the advances in nanofabrication that is owed to the explosive rise of miniaturized semiconductor electronics. The resulting field, nanophotonics, has opened a vast design space for applied researchers and required revisiting some of the oldest problems and assumptions of optical physics. 
%Polarization, meaning, in the context of light, the direction of oscillation of the electromagnetic field in space, is a particularly malleable property of light that can be used to shape and direct wave fronts, to measure and control light-matter interactions, and to encode information. It remains an underexplored and underutilized feature of nature, though the new methods of nanophotonics can harness its potential to a much greater extent than any previous optical technology platform. This thesis explores some aspects of the role light’s polarization plays at the interface of optics and nanotechnology. In particular, it will touch upon the way polarization may be used to control the generation of optical nearfields, how the polarization structure of evanescent waves leads to unusual optical forces, and how nanoscale polarization-transformations enable a new class of polarization-sensitive optical elements. It will also show how nanophotonics may address the problem of measuring polarization based on a new polarimeter architecture.
