shows that mechanical forces are produced by plasmonic excitation on nanowires, a phenomenon that has been widely overlooked and points to a new family of plasmonically-driven processes. On a fundamental level, this chapter investigates the action of the Lorentz force on a free electron gas that is bound to the surface of a nanowire. Appreciable mechanical forces are produced by wavelength illuminations between longitudinal and transverse absorption resonances via the excitation of chiral hybrid plasmon modes. The plasmonic activity is associated as the underlying mechanism for nanowire rotation, which explains prior experimental results. The presence of chiral hybrid plasmon modes yield larger net translation and torque forces than either transverse or longitudinal plasmon modes. The asymmetric plasmon behavior affects the complex nonlinear dynamics of plasmonic nonspherical nanoparticles in fluids.
