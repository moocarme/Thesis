investigates magneto-plasmonics, the ability to perturb a plasmonic response with magnetic fields, in gold nanoparticles. Magneto-plasmonics is generally associated with ferromagnetic-plasmonic materials because such optical-derived responses from nonmagnetic materials alone are considered weak, however, this chapter shows the measurable magneto-optical responses in gold nanoparticles. This chapter shows that there exists a transition between linear and nonlinear magneto-optical behaviors in gold nanocolloids that is observable at ultra-low illumination intensities and direct-current magnetic fields. The response is attributed to polarization-dependent nonzero-time-averaged plasmonic loops, vortex power flows, and nanoparticle magnetization. This chapter identifies significant mechanical effects that arise via magnetic-dipole interactions, exhibited via changes in the transmission spectra.
