\section{What has been done}
The work carried out as part of this thesis has discussed many aspects on the optical properties and manipulation of metallic nanoparticles. The light-matter interactions studied here range from the magneto optical effects that can be exhibited in non ferromagnetic nanoparticles to the spatial encoding of information via fractal architectures, and ultimately contribute to the next generation of metamaterials and metasurfaces. To reveal the full potential of optical manipulation devices such as metasurfaces they will have to manufactured at large scale using bottom-up techniques. While top-down techniques, such as electron-beam lithography, offer excellent nanometer-precision of the arrangement of individual features, typically only small areas can be patterned at a time ($<$1cm$^2$), and suffer from limitations such as the need for expensive machines and long fabrication time. 

The work in this thesis studies the optically generated forces associated with metasurfaces and metamaterials, as well as their core building blocks, individual metallic nanoparticles, that may make bottom-up assembly possible in the future. Also studied in this thesis are applications to metasurfaces which include increased and robust data transmission rates, as well as a demonstration of their versatility via the fabrication of a chiral metasurface by the arrangement of nonchiral sub-structures.

The light-matter interactions studied here are derived from plasmonic excitations that occur when light is incident on metallic nanoparticles. Plasmonic interactions are popular for a variety of factors including high local field confinement, and therefore enhancement of nonlinear phenomena [\cite{Novotny2011, Kauranen}]. Popularity of plasmonic materials also stems from their sensitivity to the optical properties the metal and dielectric materials that compose the interface of which plasmons are created [\cite{Homola}]. Further exploration into exploiting these features of plasmonic have been studied here. 

In chapter 2 magneto-optical effects are observed in gold nanoparticles. The nanoparticles are illuminated by circularly-polarized light which generate plasmon-assisted solenoidal currents that flow on the surface of the nanoparticles. According to Faraday's law it is these solenoidal currents that form the basis of magnetization of the nanoparticles, whose effects are observed experimentally via the extinction spectra.

In chapter 3 the Lorentz force is investigated when nanowires are illuminated with plane waves at oblique incidences.The plasmonic modes that are excited on the nanowire induce forces and torques that explain the rocking and spinning motion that other experimental scientists have observed, yet have been unexplained.

In chapter 4 chirality is shown in a metasurface composed of non-chiral sub structures, namely rod-shaped apertures.  The chirality is present sue to interactions between neighboring substructures that is derived from the Li\'{e}nard-Wiechert potential, and is observed experimentally in the polarization through the metasurface. The polarization exhibits optical rotation and circular dichroism at normal incidence\textemdash the signature of chiral phenomena. Developing chiral media is important for applications where either accurate control over the polarization is desired or for the development of a new generation of polarization detectors [\cite{Mueller:16}].

Chapter 5 shows that information is encoded robustly when designed and transmitted with fractal architecture. Self-similar transmission profiles are less sensitive to intermediate-obstacle signal blocks and are regenerated easily without distortion of the original image. Applications of this work may be relevant to any large-scale data transmission platforms.


\section{What is left to do}
One of the limitations listed of nanofabrication is the necessity for extremely expensive machines that make research inaccessible to many. One of the future goals of the research studied in this thesis is to develop new methods to self-assemble structures at the nanoscale. Understanding how Lorentz forces are generated at the nanoscale is unintuitive, yet fundamental to fabricating consistent, uniform nanostructures at scale. Further explorations into this topic may look into further methods to validate the magnitudes, directions, and locations of Lorentz forces when light is applied to nanostructures. 

Challenges in the experimental verification occur in a number of ways, from the optical trapping of nanostructures, to confinement of light at the orders similar to the wavelength. Regardless, the highly diverse nature of plasmonics presents an exciting future, where, for example the development of ultrafast computing, active plasmonic devices or biochemical applications.

Control over the polarization has been explored here through the use of a chiral metasurface. It is a goal of scientists to probe the nature of materials via various optical characterization methods. This work began by simply observing the scattering and absorption properties [\cite{Mie}], this method later gave rise to spectrophotometry. Optical characterization techniques since grown to include methods such as Fourier transform infra-red spectroscopy [\cite{Chan:2016}], ellipsometry [\cite{Theeten:1981}] and surface enhanced Raman spectroscopy [\cite{Wang:2015}]. To this end it is the goal of scientists to develop materials that are able to efficiently modify the polarization fields of light to probe a materials properties such as the structure or composition. Metasurfaces are one way of achieving efficient polarization control, they modify the polarization over length scales that are less than the wavelength. For comparison, traditional polarization optics need length scales around 3 orders of magnitude greater than the wavelength, such as polarizers, and quarter wave-plates, making metasurfaces very attractive in this regard.

One area of improvement that can be made in the development of future metasurfaces is reconfigurability. Similar to how a spatial light modulator is able to manipulate the phase, the same can be achieved with metasurfaces. The work presented as part of this thesis lays out methods to optically manipulate either metallic nanoparticles (chapters 2, 3) or to manipulate the polarization (chapter 4), from metallic nanostructures reminiscent of nanoparticles easily available (gold nanorods). It is conceivable that future applications of the work studied here may combine the two, optically arranging metallic nanoparticles on a large scale using Lorentz forces derived from the plasmonic interaction, into some configuration, for example exhibiting a linear phase gradient [\cite{Aieta}], or even the fractal geometries explored in chapter 5. The metasurface could demonstrate its function, manipulating the polarization of light. Following, the metasurface may be reconfigured for another purpose or function. Challenges in this area will involve uniform fabrication\textemdash any irregularities may give rise to structural correlations that lead to unwanted effects, or nonlocal effects at the nanoscale. Challenges may also be faced in having to overcome any optical losses associated with metallic nanoparticles, where dielectric materials may be more desirable. Work in this area may fall under the category of aperiodic metamatrial research.

The complete control over structures at the nanoscale is highly attractive and has the potential to usher an era of large-scale manufacturing of materials to probe the fundamentals properties of materials. Metallic nanoparticles can act as the building blocks of metasurfaces and metamaterials, not only can their size, shape, and composition be modified to produce the desired optical properties, but their spatial arrangement can lead to more complicated and precise control over light, from intensity to colour to polarization. This thesis represents a few small progressions to these goals.